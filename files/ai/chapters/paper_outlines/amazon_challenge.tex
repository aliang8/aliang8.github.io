\begin{itemize}
  \item Amazon Picking Challenge tests ability of robotic system to fulfill orders by picking items from a warehouse shelf
  \item Build tightly integrated systems and modularize the system by breaking it down into simpler subproblems
  \item In manipulation it is important to consider alternative embodiments
  \item Integrate planning with feedback from physical interactions, interactions help reduce uncertainty
  \item Finding general solutions is desirable but may be infeasible. Try to search for reasonable and useful assumptions to simplify problem
  \item Challenges: narrow bins and objects were barely visible and partially obstructed, floor made of reflective metal, poor lighting conditions
  \item Use joint- and task- space feedback controllers
  \item Use a mobile base to allow the arm to reposition itself to generate easier grasps, but increased the dimensionality of the configuration space
  \item Simple end-effector that using a suction cup, can pick up most objects, thin shape reduces need to consider complex collision avoidance
  \item Proper embodiment simplifies the overall solution
  \item Use a hybrid automaton where states correspond to a feedback controller and state transitions are triggered by sensor events
  \item Most of the failure cases occurered because of perception inability to discriminate objects properly
  \item There is recurring underlying structure in robotics problems, making suitable assumptions helps alleviate difficulties of general purpose solutions
  \item Adding explicit knowledge about physics makes problem more tractable
\end{itemize}