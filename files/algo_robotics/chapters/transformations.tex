\documentclass[../main.tex]{subfiles}

\begin{document}

\section{Homogenous Transforms}
\subsection{Conventions}
  \begin{itemize}
    \item Objects are abstracted by a set of axes fixed to the body, called coordinate frames.
    \item Points possess position but not orientation. Rigid bodies possess both position and orientation.
    \item Mechanics is about relation between two objects.
    \begin{itemize}
      \item a is "r-related" to b is: $r_{a}^{b}$.
      \item velocity (v) of a robot (r) relative to (e): $v_{r}^{e}$
      \item "r" is not a property of a. "r" is a property of a \textit{relative} to b
    \end{itemize}
    \item Relationship is directional and asymmetric: $r_{a}^{b} \neq r_{b}^{a}$
    \item Vectors of physics are coordinate system independent.
    \item Vectors of linear algebra are coordinate system dependent.
    \item Subscripts denote the object frame possessing the quantity: $v_{wheel}$ - velocity of the wheels
    \item Superscripts denote the coordinate system within which the quantity is expressed: $v_{wheel}^{world}$ - velocity of the wheel w.r.t the world
  \end{itemize}

  \subsection{Definitions}
  \begin{itemize}
    \item Affine Transforms: most general linear transform
    \begin{itemize}
      \item $\begin{bmatrix} x_{2} \\ y_{2} \end{bmatrix} = \begin{bmatrix} r_{11} & r_{12} \\ r_{21} & r_{22} \end{bmatrix} \begin{bmatrix} x_{1} \\ y_{1} \end{bmatrix} + \begin{bmatrix} t_{2} \\ t_{2} \end{bmatrix}$
      \item Can be used for translation, rotation, scale, reflections, and shears
      \item Preserves linearity but not distance (hence not areas or angles)
    \end{itemize}

    \item Homogeneous Transforms: $t_{1} = t_{2} = 0$
    \begin{itemize}
      \item $\begin{bmatrix} x_{2} \\ y_{2} \end{bmatrix} = \begin{bmatrix} r_{11} & r_{12} \\ r_{21} & r_{22} \end{bmatrix} \begin{bmatrix} x_{1} \\ y_{1} \end{bmatrix} + \begin{bmatrix} 0 \\ 0 \end{bmatrix}$
      \item Can be used for rotation, scale, reflections, and shears (\textbf{not translation})
      \item Preserves linearity but not distance (hence not areas or angles)
    \end{itemize}

    \item Orthogonal Transforms: Same as homogeneous transforms, buts
    \begin{itemize}
      \item $\begin{bmatrix} x_{2} \\ y_{2} \end{bmatrix} = \begin{bmatrix} r_{11} & r_{12} \\ r_{21} & r_{22} \end{bmatrix} \begin{bmatrix} x_{1} \\ y_{1} \end{bmatrix} + \begin{bmatrix} 0 \\ 0 \end{bmatrix}$
      \item \begin{align*} r_{11}r_{12} + r_{21}r_{22} = 0 \\ r_{11}r_{11} + r_{21}r_{21} = 1 \\ r_{12}r_{12} + r_{22}r_{22} = 1 \end{align*}
      \item Pairwise dot product of columns in R must equal 0. Norm of each column must be 1.
      \item Can be used for rotation and reflections.
      \item Preserves linearity \underline{AND} distance.
    \end{itemize}

    \item Rotation Matrix: Same as orthogonal transforms, but
    \begin{itemize}
      \item $\begin{bmatrix} x_{2} \\ y_{2} \end{bmatrix} = \begin{bmatrix} r_{11} & r_{12} \\ r_{21} & r_{22} \end{bmatrix} \begin{bmatrix} x_{1} \\ y_{1} \end{bmatrix} + \begin{bmatrix} 0 \\ 0 \end{bmatrix}$
      \item \begin{align*} r_{11}r_{12} + r_{21}r_{22} = 0 \\ r_{11}r_{11} + r_{21}r_{21} = 1 \\ r_{12}r_{12} + r_{22}r_{22} = 1 \\ det(R) = 1\end{align*}
      \item Can be used for rotations.
      \item Preserves linearity \underline{AND} distance.
    \end{itemize}

    \item Orientation: altitude and (heading or yaw)
    \item Pose: position and orientation
    \begin{itemize}
      \item 2D: $\begin{bmatrix} x & y & \psi \end{bmatrix}^{T}$
      \item 3D: $\begin{bmatrix} x & y & z & \theta & \phi & \psi \end{bmatrix}^{T}$
    \end{itemize}
    \item: Posture: pose plus some configuration
  \end{itemize}

  \subsection{Homogeneous Transforms}
  \begin{itemize}
    \item Pure Direction
    \item Points in 3D can be rotated, reflected, scaled and shared with 3x3 matrices but not translated.
    \item Trick: Move to 4D
    \item
    \begin{align*}
        p_{2} = p_{1} + p_{k} = \begin{bmatrix} x_{1} \\ y_{1} \\ z_{1} \\ 1 \end{bmatrix} + \begin{bmatrix} x_{k} \\ y_{k} \\ z_{k} \\ 1 \end{bmatrix} = \begin{bmatrix} 1 & 0 & 0 & x_{k} \\ 0 & 1 & 0 & y_{k} \\ 0 & 0 & 1 & z_{k} \\ 0 & 0 & 0 & 1 \end{bmatrix} \begin{bmatrix} x_{1} \\ y_{1} \\ z_{1} \\ 1 \end{bmatrix} = trans(p_{k})p_{1}
    \end{align*}
    \item Operators
    \begin{itemize}
        \item
        $trans(u, v, w) = \begin{bmatrix} 1 & 0 & 0 & u \\ 0 & 1 & 0 & v \\ 0 & 0 & 1 & w \\ 0 & 0 & 0 & 1 \end{bmatrix}$
        \item
        $rot_{x}(\theta) = \begin{bmatrix} 1 & 0 & 0 & 0 \\ 0 & cos(\theta) & -sin(\theta) & 0 \\ 0 & sin(\theta) & cos(\theta) & 0 \\ 0 & 0 & 0 & 1 \end{bmatrix}$
        \item Operating on a point v.s. operating on a direction
    \end{itemize}
    \item The columns of the identity HT can be considered to represent the coordinate frame itself.
    \item Homogeneous Transforms are both operators and frames. They can be both the things that operate on other things and things operated upon.
  \end{itemize}

  \subsection{Euler Angles}
  \begin{itemize}
    \item Can define rotation relative to axes of a frame
    \item Euler angles have trouble rotating about two or more axes
    \item Many euler angles map to one rotation (gimbal lock)
  \end{itemize}

  \subsection{Quaternions}
\end{document}