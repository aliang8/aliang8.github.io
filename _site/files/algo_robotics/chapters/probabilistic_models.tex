\documentclass[../main.tex]{subfiles}

\begin{document}
\section{Probability Basics}

\subsection{Discrete Random Variables}
    \begin{itemize}
      \item $X$ denotes a random variable and it can take on a countable number of values in $\{x_{1}, x_{2}, ..., x_{n}\}$
      \item $P(X=x_{i})$ is the probability that the random variable X takes on value $x_{i}$
      \item $P(...)$ is called the \textbf{probability mass function}
      \item e.g. $P(Room) = \langle 0.7, 0.2, ..., 0.02 \rangle $
    \end{itemize}

\subsection{Continuous Random Variables}
    \begin{itemize}
      \item $X$ takes on a value in the continuum
      \item $P(X = x)$ is the $probability density function$
      \item $P(x \in (a,b)) = \int_{a}^{b}P(x)dx$
    \end{itemize}

\subsection{Axioms of Probability Theory}
    \begin{itemize}
      \item $0 \leq P(a) \leq 1$
      \item $P(true) = 1$ and $P(false) = 0$
      \item $P(a \vee b) = P(a) + P(b) - P(a \wedge b)$
    \end{itemize}

\subsection{Joint and Conditional Probability}
    \begin{itemize}
      \item $P(X = x \wedge Y = y) = P(x, y)$
      \item If $X$ and $Y$ are \textbf{independent} then $P(x,y) = P(x)P(y)$
      \item $P(x|y) = \frac{P(x,y)}{P(y)}$
      \item $P(x,y) = \frac{P(x|y)}{P(y)}$
      \item If $X$ and $Y$ are \textbf{independent} then $P(x|y) = P(x)$
      \item $P(x,y|z) = P(x|z)P(y|z)$ means that x and y are \textbf{conditionally independent}
      \item If I know z, I don't need to know x to compute the probability of y.
    \end{itemize}

\subsection{Law of Total Probability (Discrete)}
    \begin{itemize}
      \item $\sum_{x}P(x) = 1$
      \item $P(x) = \sum_{y}P(x,y) = \sum_{y}P(x|y)P(y)$
    \end{itemize}

\section{Bayes Rule}
    \begin{itemize}
      \item $P(x|y) = \frac{P(y|x)P(x)}{P(y)} = \frac{likelihood * prior}{evidence}$
      \item Usually $P(y)$ is difficult to compute, so use normalization trick
      \item $P(x|y) = \eta P(y|x)P(x)$ where $\eta = \frac{1}{\sum_{x \in X} P(y|x)P(x)}$
    \end{itemize}

\subsection{Casual and Diagnostic Reasoning}
    \begin{itemize}
      \item Suppose a robot wants to determine probability of a door being open
      \item It obtains measurement $z$. What is the $P(open|z)$
      \item $P(open|z)$ is \textbf{diagnostic} and $P(z|open)$ is \textbf{causal}
    \end{itemize}

\subsection{Conditional Independence Example}
    \begin{itemize}
      \item Consider three variables: $RobotLocation, GPSEstimate, LandmarkEstimate$
      \item $GPSEstimate$ and $LandmarkEstimate$  are NOT independent, $P(GPSEstimate|LandmarkEstimate) != P(GPSEstimate)$
      \item $GPSEstimate$ and $LandmarkEstimate$ are conditionally independent given $RobotLocation$
      \item If I know the robot's location, then I can compute the landmark estimate without knowing the GPS estimate
    \end{itemize}

\section{Bayes Net}
    \begin{itemize}
      \item Encode conditional independence relationships in a Bayes Net. Used to describe cause-effect relationships
      \item Directed and acyclic graph
      \begin{itemize}
        \item Nodes represent random variables
        \item Edges represent conditional dependencies
        \item Nodes that are not connected are conditionally independent of each other
        \item Node is associated with a probability function $P(X_{i} | Parents(X_{i}))$, this is defined by a conditional probability table (CPT)
      \end{itemize}
      \item Inference ....
    \end{itemize}

\subsection{Markov Random Fields}
    \begin{itemize}
      \item Graph is undirected and may be cyclic
    \end{itemize}

\subsection{Conditional Random Fields}
    \begin{itemize}
      \item Undirected graphical model whose nodes can be divided into exactly two disjoint sets:
      \begin{itemize}
        \item X: the input variables
        \item Y: the observed and output variables
      \end{itemize}
      \item Used to model the conditional distribution: $P(Y|X)$
      \item CRFs can be used for object recognition and image segmentation
    \end{itemize}

\section{Learning a probabilistic model}
    \begin{itemize}
      \item
    \end{itemize}

\end{document}