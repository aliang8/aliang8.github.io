\documentclass[../main.tex]{subfiles}

\begin{document}
    \begin{itemize}
        \item Motion Planning is the automatic generation of motion / path for a robot that does not collide with obstacles.
        \item Path planning is global search for path to goal whereas obstacle avoidance ("local navigation") is a reactive method
        \begin{itemize}
            \item Exact algorithms
            \begin{itemize}
                \item Either find a solution or prove one doesn't exist
                \item Computationally expensive
                \item Unsuitable for high-dimensional spaces
            \end{itemize}
            \item Discrete search
            \begin{itemize}
                \item Divide space into grid, use A* search
                \item Unsuitable for high-dimensional spaces
            \end{itemize}
            \item Sampling-based planning
            \begin{itemize}
                \item Sample the C-space, construct path from samples
                \item Good for high-dimensional spaces
                \item Weak completeness and optimality guarantees
            \end{itemize}
        \end{itemize}
    \end{itemize}

    \section{Methods}
    \begin{itemize}
        \item Visibility graph
        \begin{itemize}
            \item Continuous representation (configuration space formulation)
            \item Discretization (random sampling)
            \item Graph searching (BFS, DFS, A*)
        \end{itemize}
        \item A visibility graph is a graph such that
        \begin{itemize}
            \item nodes: $q_{init}, q_{goal}$ or obstacle vertex
            \item edges: edge exists between nodes $u$ and $v$ if the line segment between $u$ and $v$ is an obstacle edge or does not intersect the obstacles
        \end{itemize}
        \item Algorithm
        \item Computational Efficiency
        \item Cell decomposition: decompose free space into \underline{simple} cells and represent connectivity of free space by adjacency graph of these cells
        \begin{itemize}
            \item
        \end{itemize}
        \item Potential field: define potential function over free space that has global minimum at goal and follow the steepest descent of the potential function
    \end{itemize}

    \section{Configuration Space}
    \begin{itemize}
        \item Open set - a set with no boundary. Every point in the set has an open neighborhood which is also in the set.
        \item Closed set - a set with a boundary. A closed set is a complement of some open set and vice versa.
        \item A set $X$ is called a topological space if there is a collection of open subsets of $X$ such as
        \begin{itemize}
            \item the union of any number of open sets is an open set
            \item the intersection of a finite number of open sets is an open sets
            \item both $X$ and $\emptyset$ are open sets
        \end{itemize}
        \item Two topological spaces $X$ and $Y$ are \textbf{homeomorphic} if there is a bijective function $f: X \rightarrow Y$ and both $f$ and $f^{-1}$ are continuous.
        \item Homeomorphisms can not add or remove holes.
        \item Common Topological Spaces: the real numbers ($\mathbb{R}^{1}$), the unit circle ($\mathbb{S}^{1}$)
        \item Can make more complex spaces using the Cartesian product.
        \item $\mathbb{R}^{1} \times \mathbb{S}^{1}$ = a hollow cylinder
        \item The \textbf{configuration} of a moving object is a specification of the position of every point on the object
        \begin{itemize}
            \item A configuration $q$ is usually expressed as a vector of the DOF of the robot
        \end{itemize}
        \item The \textbf{configuration space} $C$ is the set of all possible configurations. Usually this is a topological space. A configuration $q$ is a point in $C$.
        \item The \textbf{dimension of a configuration space} is the minimum number of DOF needed to specify the configuration of the object completely.
        \item An \textbf{articulated} object is a set of rigid bodies connected by joints.
        \item A \textbf{path} in $C$ is a continuous curve connecting two configurations $q_{start}$ and $q_{goal}$.
        \item A \textbf{trajectory} is a path parameterized by time.
        \begin{itemize}
            \item A configuration $q$ is collision-free if the robot placed at $q$ does not intersect any obstacles in the workspace.
            \item The free space $C_{free}$ is a subset of $C$ containing all free configurations.
            \item A configuration space obstacle $C_{obs}$ is a subset of $C$ that contains all configurations where the robot collides with workspace obstacles or with itself.
        \end{itemize}
        \item Minkowski sum [insert image]
    \end{itemize}

    \section{Sampling-based Planning}
    \begin{itemize}
        \item How do we plan in \textbf{high-dimensional} C-spaces?
        \item Exact methods either find a solution of prove none exists
        \item They require computing C-space obstacles which are very computationally expensive!
        \item Discrete search run-time and memory requirements are sensitive to branching factors (\# of successors)
        \item Number of sessions depends on dimension, n-dimensional 8-connected space has $3^{n} - 1$ successors
        \item In sampling-based planning, instead of systematically-discretizing the C-space, take samples in the C-space and use them to construct a graph.
        \item Advantages
        \begin{itemize}
            \item Don't need to discretize
            \item Don't need to explicitly represent C-space
            \item Easy to sample high-dimensional spaces
        \end{itemize}
        \item Disadvantages
        \begin{itemize}
            \item Probability of sampling an area depends on the area's size, hard to sample narrow passages
            \item No guarantees on completeness / optimality
        \end{itemize}
        \item Prrobabilistic Roadmap (PRM) - multi-query algorithms because roadmap can be reused if environment and robot haven't changed between queries
        \begin{itemize}
            \item Build a roadmap of the space from sampled points and search roadmap to find a path
        \end{itemize}
        \begin{enumerate}
            \item "Learning" Phase
            \begin{enumerate}
                \item Construction step
                \begin{enumerate}
                    \item Build roadmap by sampling random free configurations and connect them using a fast local planner
                    \item Store configurations as nodes in a graph
                    \item Edges of graph are paths between nodes found by local planner
                \end{enumerate}
                \begin{itemize}
                    \item Need a distance metric to define "nearest": $D(q_{1}, q_{2})$, use Euclidean distance
                    \item Naive NN can be slow with 1000s of nodes, so use \textbf{kd-tree} to store nodes and do NN queries
                    \item Kd-tree is a data structure that recursively divides the space into bins that contains points (like Oct-tree) and nearest neighbor searches through bins to find nearest point
                    \item Local planner can be anything, but must be fast because it is called many times by the algorithm
                    \item Easiest way is just to connect points using a straight line and check whether the line is collision free
                \end{itemize}
                \item Expansion step
                \begin{itemize}
                    \item You can have disconnected components that should be connected
                    \item Expansion uses heuristics to sample more nodes to connect disconnected components
                    \item No "right" way, this step is environment dependent
                \end{itemize}
            \end{enumerate}
            \item Query Phase
            \begin{itemize}
                \item Given start $q_{s}$ and goal $q_{g}$. Connect them to the roadmap using a local planner.
                \item Then search the graph G to find the shortest path between $q_{s}$ and $q_{g}$ using A*, Dijkstra's, etc.
            \end{itemize}
            \begin{algorithm}[H]
                \SetAlgoLined
                \For{i = 0 to maxiterations} {
                    pick two points $q_{1}$ and $q_{2}$ on the path randomly\;
                    try to connect them with a line segment\;
                    if successful, replace path between $q_{1}$ and $q_{2}$ with the line segment\;
                }
                \caption{Path shortening / smoothing}
            \end{algorithm}
            \item PRM Failure Modes
            \begin{itemize}
                \item Cannot connect $q_{s}$ and $q_{g}$ to any nodes in the graph
                \item Cannot find a path in the graph but path is possible
            \end{itemize}
            \item PRM issues
            \begin{itemize}
                \item Uniform random sampling misses narrow passages
                \item Exploring whole space, but all we want is a path
            \end{itemize}
            \item Sampling strategies
            \begin{itemize}
                \item Gaussian sample
                \item Bridge sample
                \begin{itemize}
                    \item Sample a $q_{1}$ in collision
                    \item Sample a $q_{2}$ in neighborhood of $q_{1}$ with some prob distribution
                    \item If $q_{2}$ is in collision, get the midpoint of ($q_{1}, q_{2}$)
                    \item Check if midpoint is in collision and if not add it as a node
                \end{itemize}
            \end{itemize}
        \end{enumerate}
        \item Rapidly-exploring Random Trees (RRTs) : single-query method
        \begin{itemize}
            \item Build a \textbf{tree} instead of a graph**
            \item The tree grows in $C_{free}$
            \item Like PRM captures some connectivity, but unlike PRM it only explores what is connected to $q_{start}$
            \item RRT Goal Biasing
            \begin{itemize}
                \item Bias RRTs towards goal to produce a path
                \item When generating a random sample, with some probability pick the goal instead of random node
            \end{itemize}
            \item RRT Extension Types
            \begin{itemize}
                \item RRT-Extend: Take one step towards a random direction
                \item RRT-Connect: Step towards random sample until it is either reached or you hit an obstacle
                \item BiDirectional RRTs: grow tree from both start and goal
                \item RRT produces bad paths, must perform path smoothing (ALWAYS)
            \end{itemize}
        \end{itemize}
        \begin{algorithm}[H]
            \SetAlgoLined
            $q_{node} = q_{start}$\;
            \For{i = 1 to num\_samples} {
                $q_{rand}$ = sample near $q_{node}$\;
                Add edge $e = (q_{rand}, q)$ if collision-free\;
                $q_{node}$ = pick random node of tree\;
            }
            \caption{Naive Tree algorithm}
        \end{algorithm}
        \begin{algorithm}[H]
            \SetAlgoLined
            T.init($q_{init}$)\;
            \For{k = 1 to K} {
                $q_{rand}$ = random\_config()\;
                extend(T, $q_{rand}$)\;
            }
            \caption{Build RRT}
        \end{algorithm}
    \end{itemize}

    \section{Nonholonomic Planning}
    \begin{itemize}
        \item \textbf{Holonomic constraints} depend only on configuration
        \item $F(q,t) = 0$
        \item These have to be bilateral constraints (no inequalities)
        \item Example: kinematics of a unicycle
        \begin{itemize}
            \item Can move forward and backward
            \item Can rotate about the wheel center
            \item Can't move sideways
        \end{itemize}
        \item \textbf{Non-holonomic} constraints are non-integrable. Thus they must contain derivatives of configuration. Sometimes called differential constraints.
        \item State space (configuration + velocity) vs control space (speed or acceleration, steering)
        \item Simple Car
        \begin{itemize}
            \item Non-holonomic constraint: $-\dot{x}sin(\theta) + \dot{y}cos(\theta) = 0$
            \item Essentially means you can't move sideways
        \end{itemize}
        \item Two-point Boundary Value Problem (BVP): find a control sequence to take system from state $x_{i}$ and $x_{g}$ while obeying kinematic constraints
        \item Discrete Planning Option 1: Sequencing primitives
        \begin{itemize}
            \item Discretize control space into primitives (pick steering angles, accelerations, velocities)
            \item Disadvantage: losing full continuous completeness and discontinuous curvature
            \item Choice of primitives affects completeness, optimality, and speed
        \end{itemize}
        \item Discrete Planning Option 2: State Lattice
        \begin{itemize}
            \item Pre-compute state lattice
            \item Two methods to get lattice:
            \begin{itemize}
                \item Forward: using motion primitives
                \item Inverse: Use BVP solvers to find trajectories between states
            \end{itemize}
            \item Impose continuity constraints at graph vertices
            \item Search state lattice like any graph
        \end{itemize}
        \item Sampling-based Planning
        \begin{itemize}
            \item Building state lattice is impractical in high dimensions
            \item We are now sampling state space, not C-space!
            \item Challenges
            \begin{itemize}
                \item Dimension of space is doubled
                \item Moving between points is harder
                \item Distance metric is unclear (worst problem)
            \end{itemize}
            \item RRT Non-holonomic Planning
            \begin{itemize}
                \item
            \end{itemize}
        \end{itemize}
    \end{itemize}
\end{document}